%% BioMed_Central_Tex_Template_v1.06
%%                                      %
%  bmc_article.tex            ver: 1.06 %
%                                       %

%%IMPORTANT: do not delete the first line of this template
%%It must be present to enable the BMC Submission system to
%%recognise this template!!

%%%%%%%%%%%%%%%%%%%%%%%%%%%%%%%%%%%%%%%%%
%%                                     %%
%%  LaTeX template for BioMed Central  %%
%%     journal article submissions     %%
%%                                     %%
%%          <8 June 2012>              %%
%%                                     %%
%%                                     %%
%%%%%%%%%%%%%%%%%%%%%%%%%%%%%%%%%%%%%%%%%


%%%%%%%%%%%%%%%%%%%%%%%%%%%%%%%%%%%%%%%%%%%%%%%%%%%%%%%%%%%%%%%%%%%%%
%%                                                                 %%
%% For instructions on how to fill out this Tex template           %%
%% document please refer to Readme.html and the instructions for   %%
%% authors page on the biomed central website                      %%
%% http://www.biomedcentral.com/info/authors/                      %%
%%                                                                 %%
%% Please do not use \input{...} to include other tex files.       %%
%% Submit your LaTeX manuscript as one .tex document.              %%
%%                                                                 %%
%% All additional figures and files should be attached             %%
%% separately and not embedded in the \TeX\ document itself.       %%
%%                                                                 %%
%% BioMed Central currently use the MikTex distribution of         %%
%% TeX for Windows) of TeX and LaTeX.  This is available from      %%
%% http://www.miktex.org                                           %%
%%                                                                 %%
%%%%%%%%%%%%%%%%%%%%%%%%%%%%%%%%%%%%%%%%%%%%%%%%%%%%%%%%%%%%%%%%%%%%%

%%% additional documentclass options:
%  [doublespacing]
%  [linenumbers]   - put the line numbers on margins

%%% loading packages, author definitions

\documentclass[twocolumn]{bmcart}% uncomment this for twocolumn layout and comment line below
%\documentclass{bmcart}

%%% Load packages
\usepackage{amsthm,amsmath}
\usepackage{siunitx}
\usepackage{mfirstuc}
%\RequirePackage{natbib}
\usepackage[colorinlistoftodos]{todonotes}
\RequirePackage{hyperref}
\usepackage[utf8]{inputenc} %unicode support
%\usepackage[applemac]{inputenc} %applemac support if unicode package fails
%\usepackage[latin1]{inputenc} %UNIX support if unicode package fails
\usepackage[htt]{hyphenat}

\usepackage{array}
\newcolumntype{L}[1]{>{\raggedright\let\newline\\\arraybackslash\hspace{0pt}}p{#1}}

%%%%%%%%%%%%%%%%%%%%%%%%%%%%%%%%%%%%%%%%%%%%%%%%%
%%                                             %%
%%  If you wish to display your graphics for   %%
%%  your own use using includegraphic or       %%
%%  includegraphics, then comment out the      %%
%%  following two lines of code.               %%
%%  NB: These line *must* be included when     %%
%%  submitting to BMC.                         %%
%%  All figure files must be submitted as      %%
%%  separate graphics through the BMC          %%
%%  submission process, not included in the    %%
%%  submitted article.                         %%
%%                                             %%
%%%%%%%%%%%%%%%%%%%%%%%%%%%%%%%%%%%%%%%%%%%%%%%%%


%\def\includegraphic{}
%\def\includegraphics{}

%%% Put your definitions there:
\startlocaldefs
\endlocaldefs


%%% Begin ...
\begin{document}

%%% Start of article front matter
\begin{frontmatter}

\begin{fmbox}
\dochead{Report from 2015 OHBM Hackathon (HI)}

%%%%%%%%%%%%%%%%%%%%%%%%%%%%%%%%%%%%%%%%%%%%%%
%%                                          %%
%% Enter the title of your article here     %%
%%                                          %%
%%%%%%%%%%%%%%%%%%%%%%%%%%%%%%%%%%%%%%%%%%%%%%

\title{DIPY: Brain tissue classification}
\vskip2ex
\projectURL{Project URL: \url{https://github.com/villalonreina/dipy/tree/pve}}

\author[
addressref={aff1},
corref={aff1},
email={julio.villalon@ini.usc.edu}
]{\inits{JEVR} \fnm{Julio E.} \snm{Villalon-Reina}}
\author[
addressref={aff2},
%
email={garyfallidis@gmail.com}
]{\inits{EG} \fnm{Eleftherios} \snm{Garyfallidis}}

%%%%%%%%%%%%%%%%%%%%%%%%%%%%%%%%%%%%%%%%%%%%%%
%%                                          %%
%% Enter the authors' addresses here        %%
%%                                          %%
%% Repeat \address commands as much as      %%
%% required.                                %%
%%                                          %%
%%%%%%%%%%%%%%%%%%%%%%%%%%%%%%%%%%%%%%%%%%%%%%

\address[id=aff1]{%
  \orgname{Imaging Genetics Center, USC Stevens Neuroimaging and Informatics Institute, 
  Keck School of Medicine of USC, University of Southern California},
  \city{Marina del Rey},
  \street{4676 Admiralty Way},
  \postcode{90292},
  \postcode{California},
  \cny{USA}
}
\address[id=aff2]{%
  \orgname{Département d'informatique, Université de Sherbrooke},
  \city{Sherbrooke},
  \street{2500 boulevarde de l'Université},
  \postcode{J1K2R1},
  \postcode{Québec},
  \cny{Canada}
}

%%%%%%%%%%%%%%%%%%%%%%%%%%%%%%%%%%%%%%%%%%%%%%
%%                                          %%
%% Enter short notes here                   %%
%%                                          %%
%% Short notes will be after addresses      %%
%% on first page.                           %%
%%                                          %%
%%%%%%%%%%%%%%%%%%%%%%%%%%%%%%%%%%%%%%%%%%%%%%

\begin{artnotes}
\end{artnotes}

%\end{fmbox}% comment this for two column layout

%%%%%%%%%%%%%%%%%%%%%%%%%%%%%%%%%%%%%%%%%%%%%%
%%                                          %%
%% The Abstract begins here                 %%
%%                                          %%
%% Please refer to the Instructions for     %%
%% authors on http://www.biomedcentral.com  %%
%% and include the section headings         %%
%% accordingly for your article type.       %%
%%                                          %%
%%%%%%%%%%%%%%%%%%%%%%%%%%%%%%%%%%%%%%%%%%%%%%

%\begin{abstractbox}

%\begin{abstract} % abstract
	
%Blank Abstract

%\end{abstract}



%%%%%%%%%%%%%%%%%%%%%%%%%%%%%%%%%%%%%%%%%%%%%%
%%                                          %%
%% The keywords begin here                  %%
%%                                          %%
%% Put each keyword in separate \kwd{}.     %%
%%                                          %%
%%%%%%%%%%%%%%%%%%%%%%%%%%%%%%%%%%%%%%%%%%%%%%

%\vskip1ex

%\projectURL{\url{https://github.com/villalonreina}}
%\projectURL{https://github.com/villalonreina}

% MSC classifications codes, if any
%\begin{keyword}[class=AMS]
%\kwd[Primary ]{}
%\kwd{}
%\kwd[; secondary ]{}
%\end{keyword}

%\end{abstractbox}
%
\end{fmbox}% uncomment this for twcolumn layout

\end{frontmatter}

%{\sffamily\bfseries\fontsize{10}{12}\selectfont Project URL: \url{https://github.com/villalonreina}}

%%% Import the body from pandoc formatted text
\section{Introduction}\label{introduction}

DMRI is used for creating visual representations of the structural
connectivity of the brain, also known as tractography. Research has
shown that using a tissue classifier can be of great benefit to create
more accurate representations of the underlying connections
\cite{Girard2014}.

The aim of this project was to implement an image segmentation algorithm in DIPY
\cite{Garyfallidis2014} for classifying the different tissue types of
the brain using structural T1 weighted images (T1-w) and diffusion MRI images (dMRI), and to 
incorporate the resulting tissue probability maps for
Anatomically-Constrained Tractography (ACT) \cite{Smith20121924}. We used Diffusion Power Maps (DPMs), which are scalar maps that are calculated from dMRI data and have a tissue contrast similar to the T1-w. By performing the tissue classification on dMRI derived scalar maps, the T1-w to dMRI registration step can be avoided. 

\section{Approach}\label{approach}

We used a Bayesian approach for the segmentation in a similar fashion than the methods 
proposed in \cite{Zhang2001} and \cite{Avants2011} by applying the Maximum-A-Posteriori (MAP)
procedure. The prior probability was modeled with Markov Random Fields
(MRF). The MRF distribution was modeled as a Gibbs distribution. We used
the Expectation Maximization (EM) algorithm to update the tissue labels
at each site and to update the parameters of the log-likelihood in all
iterations.

\section{Results}\label{results}

The first row of figure 1 shows the tissue classification on T1-w, the
initial segmentation  based on maximum likelihood and the final segmentation after 10 iterations and
beta=0.1. Beta determines the weight of the neighborhood in the MRF
model. These two parameters were tuned and validated by permuting 42 different combinations and calculating the Jaccard index between the segmentation of the proposed method against manually segmented brains from the IBSR dataset (http://www.nitrc.org/projects/ibsr). The second row of figure 1 shows the probability maps of the three main tissue classes of the brain. The top row of Figure 2 shows on the left the Diffusion Power Map (DPM), followed by its tissue classification and the streamlines from the corpus callosum reconstructed with ACT. The bottom row of figure 2 shows the tissue probability maps of the segmentation performed on a DPM. 

\begin{figure}[h!]
  \includegraphics[width=.48\textwidth]{brainseg.png}
  \caption{\label{centfig} Example segmentations on T1 images.}
\end{figure}

\begin{figure}[h!]
  \includegraphics[width=.46\textwidth]{Power_map_fig.png}
  \caption{\label{centfig} Example segmentations on Difussion Power Maps (DPM).}
\end{figure}

\section{Conclusions}\label{conclusions}

We developed a segmentation algorithm based on a Bayesian framework by
using the MAP-MRF approach and EM. The algorithm was tested on
T1-w as well as on DPMs \cite{Dell2014}. The tissue specific
probability maps from both the T1-w and the DPMs were then used for ACT. We were able to successfully run ACT with the tissue probability maps derived from the DPMs.

%%%%%%%%%%%%%%%%%%%%%%%%%%%%%%%%%%%%%%%%%%%%%%
%%                                          %%
%% Backmatter begins here                   %%
%%                                          %%
%%%%%%%%%%%%%%%%%%%%%%%%%%%%%%%%%%%%%%%%%%%%%%

\begin{backmatter}

\section*{Availability of Supporting Data}
More information about this project can be found at: \url{https://github.com/villalonreina/dipy/tree/pve}. 
%Further data and files supporting this project are hosted in the \emph{GigaScience} repository REFXXX.

\section*{Competing interests}
None

\section*{Author's contributions}
Julio E. Villalon-Reina and Eleftherios Garyfallidis performed the project and wrote the report.

\section*{Acknowledgements}
The authors would like to thank the organizers and attendees of the 2015
OHBM Hackathon. Julio E. Villalon-Reina was funded by Google Summer of Code 2015. 

  
  
%%%%%%%%%%%%%%%%%%%%%%%%%%%%%%%%%%%%%%%%%%%%%%%%%%%%%%%%%%%%%
%%                  The Bibliography                       %%
%%                                                         %%
%%  Bmc_mathpys.bst  will be used to                       %%
%%  create a .BBL file for submission.                     %%
%%  After submission of the .TEX file,                     %%
%%  you will be prompted to submit your .BBL file.         %%
%%                                                         %%
%%                                                         %%
%%  Note that the displayed Bibliography will not          %%
%%  necessarily be rendered by Latex exactly as specified  %%
%%  in the online Instructions for Authors.                %%
%%                                                         %%
%%%%%%%%%%%%%%%%%%%%%%%%%%%%%%%%%%%%%%%%%%%%%%%%%%%%%%%%%%%%%

% if your bibliography is in bibtex format, use those commands:
\bibliographystyle{bmc-mathphys} % Style BST file
\bibliography{brainhack-report} % Bibliography file (usually '*.bib' )

\end{backmatter}
\end{document}
